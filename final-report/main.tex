\documentclass{final_report}
\usepackage[backend=biber,
            style=numeric,
            sorting=none,
            backref=true,
            language=british]{biblatex} %create bibliography
\usepackage{graphicx}
\usepackage{lastpage}
\usepackage{verbatim}
\usepackage{fancyvrb}
\usepackage[shortlabels]{enumitem} %change list formatting
\usepackage[pdftex,
            colorlinks,
            breaklinks=true,
            linkcolor={black},
            citecolor={black},
            urlcolor={blue!60!black},
            pdfauthor={George Honeywood},
            pdftitle={Final Year Project --- Final Report}]{hyperref}

% prevent long urls in the in the bibliography from overfilling (weird breaks are preferred)
\setcounter{biburlnumpenalty}{9000}
\setcounter{biburllcpenalty}{9000}
\setcounter{biburlucpenalty}{9000}

\setlist[itemize]{noitemsep, nolistsep} %make lists take up less room
\setlist[enumerate]{noitemsep}

\addbibresource{../sources.bib}

%%%%%%%%%%%%%%%%%%%%%%
%%% Input project details
\def\studentname{George Honeywood}
\def\reportyear{2022}
\def\projecttitle{Offline HTML5 Maps Application}
\def\supervisorname{Reuben Rowe}
\def\degree{BSc (Hons) in Computer Science}
\def\fullOrHalfUnit{Full Unit} % indicate if you are doing the project as a Full Unit or Half Unit
\def\finalOrInterim{Final Report} % indicate if this document is your Final Report or Interim Report

\begin{document}

\maketitle

%%%%%%%%%%%%%%%%%%%%%%
%%% Declaration

% \chapter*{Declaration}

% This report has been prepared on the basis of my own work. Where other published and unpublished source materials have been used, these have been acknowledged.

% \vskip3em

% Word Count: 

% \vskip3em

% Student Name: \studentname

% \vskip3em

% Date of Submission: 

% \vskip3em

% Signature: GH

% \newpage

%%%%%%%%%%%%%%%%%%%%%%
%%% Table of Contents
\tableofcontents\pdfbookmark[0]{Table of Contents}{toc}

%%%%%%%%%%%%%%%%%%%%%%
%%% Your Abstract here
\clearpage
\begin{abstract}
    Although online web maps are commonplace, offline maps are a valuable niche that are useful in certain situations, such as on mobile devices with limited data, or when roaming abroad. There are limited options in this space that are cross-platform, which I hope my project can resolve through the nature of it being provided as a progressive web app (PWA).

    In this project I aim to build an offline maps application based on OpenStreetMap data. The user should be able to download map data for their area of interest, then view it by panning and zooming, like a traditional online slippy map. Time permitting, I may also add additional features that require an internet connection, like routing and Wikipedia integration. Through this project I hope to learn about how you project map data into a rendered map. It will also teach me how to successfully develop a medium-sized application.
\end{abstract}

% for final report:
%%%%%%%%%%%%%%%%%%%%%%
%%% Project Spec

% \chapter*{Project specification}
% \addcontentsline{toc}{chapter}{Project specification}
% Your project specification goes here.

%%%%%%%%%%%%%%%%%%%%%%
%%% Introduction

\chapter{Introduction}

The OpenStreetMap project began in 2004, with the aim of creating a free world map~\cite{RAMM:2011}. Since then, it has met and surpassed its goals, becoming a mature, global dataset, edited by both volunteers, and corporations alike. As of late 2022, an average of 6,000 contributors edit the map daily, with more than 9 million registered in total~\cite{OSMStats}. In contrast to how cartography is traditionally carried out, OSM contributors tend to not use any specialized equipment, instead using tools like GPS and aerial imagery to create the map. The barrier of entry is low by design, making it easy for new mappers to get started. This sometimes results in inaccurate, or even vandalistic edits being made, but the community usually spots and reverts bad edits quickly.

I have personally been involved in the OpenStreetMap project since creating an account in May 2019. This was initially as I was using an OSM based map, and noticed that a new-build estate near my house was missing. Hence, \href{https://www.openstreetmap.org/changeset/70609319}{I added it}, and this led me down the rabbit hole of wanting to add all the missing features around me. Since then, I have been a regular contributor, with currently almost \href{https://www.openstreetmap.org/user/GeorgeHoneywood}{2,000 changesets} made over 341 separate days. I have also made some code contributions to the Every Door mobile editor, principally, adding support for viewing the history of element~\cite{every-door}. 

Although online OpenStreetMap-based web maps are very popular, offline maps are much less of an explored field. On the Android platform, you have a few options for offline maps, such as the venerable OsmAnd~\cite{osmand}, Organic Maps (FOSS fork of Maps.me~\cite{maps.me}), and the proprietary Magic Earth~\cite{magic-earth}. On desktop, the available options are more sparse. KDE Marble is one option~\cite{kde-marble}, but it seems like the offline experience is an afterthought. It allows you to download pre-rendered Mapnik Carto style tiles, and this is limited to above zoom level 16, so fine details are not visible (\href{https://openstreetmap.org}{openstreetmap.org} renders tiles down to zoom level 19).

This approach of downloading rendered raster tiles is also not very scalable, as there quickly becomes a huge number of images that need to be downloaded, and these images take up significant storage space. For example, I attempted to download the tiles for Cornwall, UK, in KDE Marble, and it would have had to fetch 58,928 tiles just for between zoom levels 11 and 16, equating to an estimated download size of 749 MB\@. This method is also explicitly prohibited by the OpenStreetMap Foundation's Tile Usage Policy, which states that ``In particular, downloading an area of over 250 tiles at zoom level 13 or higher for offline or later usage is forbidden.''~\cite{OSMF-tile-usage-policy}. This is because rendering map tiles is computationally expensive, and the OSMF is run with a limited budget.

Another option is the OpenStreetMap API itself~\cite{osm-api-wiki}. This is more promising for offline usage, as it allows you to download vector data that can be stored more efficiently than raster images. However, the API is not designed for this use case --- it is an interface specifically for map editor programs to use. As such, you can only download small geographical areas at a time, and the data is not stored in an optimal format for rendering. For example, complex structures, such as buildings with internal courtyards, are represented in OSM with multipolygon relations, which are difficult to parse and render correctly when using raw OSM data. An additional barrier to using the editing API is that its Terms of Use explicitly prohibit read only applications~\cite{OSM-editing-API-usage-policy}.

Cruiser follows a more promising approach~\cite{cruiser}. Instead of downloading rendered image tiles, it uses prebuilt tiled vector map files, in the Mapsforge binary format~\cite{mapsforge-format}. This is much more space efficient --- the whole of England is an 805 MB download. It is also much less difficult to host, as you don't need a powerful server to render the raster map tiles on the fly. One possible issue with this approach is that this may be too much data to store in a HTML5 web application, and this is something I will have to explore further.

Offline maps are a niche market, as desktop computers tend to always have an internet connection, and on mobile devices data is cheap enough to allow downloading some small vector or raster map tiles. Hence, the main use-case for an offline HTML5 map would be for when an internet connection is either prohibitively expensive or not available at all. This could be when roaming abroad, or when in an area without LTE coverage. As such, it makes sense to make support for mobile devices a priority, as they are the most likely to be in these situations.

From this project I hope to learn how map data actually becomes a rendered map, as currently my understanding of this process is limited. I also hope to learn more about offline HTML applications, as I think that this type of web app has lots of room to become popular in the coming years. There are lots of apps on my phone that I think could be replaced by a PWA (progressive web app).


%% these sections have basically been covered in the introduction.
%% maybe it would be good to split the introduction out.
% \section{Issues}
% \section{Motivation}

\section{Literature review}

Whilst researching my primary source for information about the OpenStreetMap project was \emph{OpenStreetMap — Using and Enhancing the Free Map of the World}, by \textcite{RAMM:2011}. This provided a good foundation of knowledge that supplemented information that I have picked up over the years from contributing to the project. Some sections were a little out of date, especially the sections on editors and tools for mappers. Notably the online editor referred to here, Potlatch, is no longer available, being superseded by iD in 2013.

For more up to date or specific information I often relied on the OpenStreetMap Wiki~\cite{osm-wiki}, which provides a helpful reference for both OSM specific information and other GIS adjacent topics. These include general information about map projections~\cite{osm-wiki-mercator}, and details about the Z/X/Y tiling scheme that is common for web maps~\cite{osm-wiki-tile-names}.

When it came to implementing the project, MDN Web Developer documentation proved invaluable~\cite{mdn}. They provide an excellent reference on how to use many web APIs, with detailed usage guides included. In particular, the information about the Canvas API~\cite{mdn-canvas-api}, and Service Workers~\cite{mdn-service-workers} was very useful, as these were technologies that I was not familiar with.

\section{Aims and objectives}

Here I will list some specific features that I would like to implement in the project. These are not fixed, and some may not be implemented, or others added in their place.

\begin{itemize}
    \item Download vector map data for a user-provided region (preferably at least as large as a UK county), allowing the user to browse the map offline
    \item Allow the user to pan and zoom the map. They should be able to zoom out to the view the full extent of the downloaded data.
    \item Provide a search functionality, using the Nominatim API
    \item Allow the user to route between two points, which could be implemented using OSRM, GraphHopper or Valhalla
    \item When an OSM element has been tagged with a reference to a Wikipedia article, it should show a description from Wikipedia
    \item Allow the user to save and name markers for later use
    \item When online, the application should allow the user to browse a map without having to first download any data. This could be done using raster tiles.
\end{itemize}

\section{Deliverables \& timeline}

Following a timeline will help my project proceed without any major unexpected delays, and will give me targets to aim for. At the beginning of term 1 I focussed on exploring any risky areas or technologies that I was unsure about. This helped answer any large questions early on in the process.

\begin{enumerate}
    \item Report on offline HTML5 technologies. Used this to discover whether it is possible to download \& store a large amount of vector map data (>100 MB) for later rendering.\label{item:deliverable:offline-html}
    \item Proof of concept basic offline HTML5 app. Used this to discover any limit of how files can be stored for offline use.\label{item:deliverable:basic-offline-app}
    \item Report on the different ways  that the program could get OpenStreetMap data. One possibility is the Mapsforge format, or it could use the OSM editing API\@. This helped mitigate risk of using a technology that is not suitable for the project.\label{item:deliverable:osm-data}
    \item Report about how map projection works. Specifically the mathematics behind projecting the data that is produced by OpenStreetMap. This ensured that I understood this key concept.\label{item:deliverable:map-projection}
    \item Proof of concept that took some way made up of latitude longitude pairs and draws a line onto a canvas.\label{item:deliverable:draw-line}
    \item Make proof of concept~\ref{item:deliverable:draw-line} interactive; allow the user to pan and zoom the map. This should be done with the scroll wheel on desktop and pinch zooming on mobile.\label{item:deliverable:pan-zoom}
    \item Proof of concept that loads some actual OSM data using the technology that I decided upon in deliverable~\ref{item:deliverable:osm-data}.\label{item:deliverable:load-osm-data}
    \item Thoroughly test the application on mobile, as this will likely be the main use case for an offline map.\label{item:deliverable:mobile-testing}
    \item Add online search functionality, using the Nominatim API.\label{item:deliverable:search}
    \item Add online routing functionality, using OSRM, GraphHopper or Valhalla.\label{item:deliverable:routing}
    \item Show point of interest information from Wikipedia, when an OSM element has been tagged to allow this.\label{item:deliverable:wikipedia}
    \item Add the ability to save and name markers for later use.\label{item:deliverable:markers}
    \item Allow the user to browse the map without first downloading data when online. This could be done using raster tiles.\label{item:deliverable:online}
\end{enumerate}

\subsection{Term one}\label{sec:term-1-plan}

\begin{itemize}
    \item \textbf{Week 3 (2022/10/03)}: Report~\ref{item:deliverable:offline-html}.
    \item \textbf{Week 4 (2022/10/10)}: Proof of concept~\ref{item:deliverable:basic-offline-app} \& report~\ref{item:deliverable:osm-data}.
    \item \textbf{Week 5 (2022/10/17)}: Report~\ref{item:deliverable:map-projection} \& proof of concept~\ref{item:deliverable:draw-line}.
    \item \textbf{Week 6--7 (2022/10/24)}: Proof of concept~\ref{item:deliverable:pan-zoom}.
    \item \textbf{Week 8--9 (2022/11/07)}: Proof of concept~\ref{item:deliverable:load-osm-data}.
    \item \textbf{Week 10--11 (2022/11/21)}: Prepare for the interim report, and presentation.
\end{itemize}

\subsection{Term two}

\begin{itemize}
    \item \textbf{Week 1--2 (2023/01/09)}: Integration of the above proof of concepts into the final program.
    \item \textbf{Week 3--4 (2023/01/23)}: Deliverable~\ref{item:deliverable:mobile-testing}.
    \item \textbf{Week 5 (2023/02/06)}: Prepare an initial draft for the final report.
    \item \textbf{Week 6--7 (2023/02/13)}: Add support for further features, such as deliverables~\ref{item:deliverable:search},~\ref{item:deliverable:routing},~\ref{item:deliverable:wikipedia},~\ref{item:deliverable:markers} and~\ref{item:deliverable:online}.
    \item \textbf{Week 8 (2023/02/27)}: Evaluate the solution so far, and decide whether to extend the project further, if time permits.
    \item \textbf{Week 9--11 (2023/03/06)}: Prepare for the final report.
\end{itemize}

\clearpage
\chapter{Research}

Here I present the research reports I conducted throughout the development process. These helped me to discover which technologies would be most appropriate for my project, and how I could implement them in my proof of concepts.

\section{Basic web technologies}

Websites are commonly made up of 3 main components: HTML, CSS and JavaScript~\cite{mdn-html}. HTML is used to create text, images, videos, and other non-interactive content. CSS is responsible for styling, colours, sizing, and other visual effects. JavaScript is used to add interactivity.

While I will need some HTML and CSS, my project is focused on creating an interactive map. Therefore, much of my work will be done in JavaScript, specifically heavily utilising the Canvas API~\cite{mdn-canvas-api}. Instead of using vanilla JavaScript, I have chosen to use TypeScript, which is a superset of JavaScript that adds a compilation/stripping step, where types are statically checked to prevent runtime type issues (this is discussed more in Section~\ref{sec:software-engineering}).

In \autoref{lst:basic-html} you can see some example HTML that does little other than create a canvas element, and load an external script. This script can then get a reference to the \texttt{map} canvas element in the DOM (e.g., with \texttt{document.querySelector("\#map")}), and use this to create and issue calls to the canvas rendering context.

\begin{lstlisting}[caption=Basic HTML to run an external script with some basic styling, language=html, label=lst:basic-html]
<!DOCTYPE html>
<html>
  <head>
    <meta name="viewport" content="width=device-width, initial-scale=1.0" />
    <title>Proof of concept: 1</title>
    <style>
      body {
        font-family: sans;
      }
    </style>
  </head>
  <body>
    <h1>Proof of concept: 1</h1>
    <canvas id="map"></canvas>
    <script src="dist/bundle.js"></script>
  </body>
</html>
\end{lstlisting}

To run your multiple JavaScript files in the browser, you can either use plain ES Modules~\cite{mdn-es-modules}, or add a bundling step, where all the code is combined into a single file. Using ES Modules in the browser allows you to avoid a build step, but has the negative that it requires \(n\) round-trips to the server. For example if say \texttt{main.js} imports \texttt{map.js} and \texttt{map.js} imports \texttt{util.js}, your browser will need to make 3 separate requests to the server, and script execution will be blocked until the last dependency has been fetched and evaluated~\cite{sitepoint-using-es-modules}. 

Bundling in its simplest form involves concatenating all the source files together into a single file, which neatly avoids the round trip problem, whilst still allowing you to store your source code in separate files. Advanced bundlers also support features like minification, where variables are renamed to be shorter, whitespace is removed, and unused functions are purged. \texttt{esbuild} is a high-performance modern JavaScript/TypeScript bundler, which is written in Go~\cite{esbuild}.

\section{Offline HTML5 applications}\label{sec:offline-html-applications}

The core of this project is being able to render a map. This, in its most basic form, means being able to take coordinates, which represent a position on the sphere-like body that is the earth, and placing these on a two-dimensional plane. This process is known as map projection~\cite[5]{canters2002small}.

OpenStreetMap uses the WGS84 coordinate reference system (EPSG:4326) to represent the positions of the nodes that make up its data~\cite{osm-wiki-wgs84}. GPS uses this CRS, and it is a popular standard~\cite{epsg.io-epsg:4326}.

If you were to naively place coordinates from the WGS84 system on to a graph, you would get a ``plate carrée'' projected map. This projection is a form of the equirectangular projection, where the standard parallel \(\phi_1 = 0\). In plate carrée, lines of longitude are straight, vertical and equidistant, and lines of latitude are similar, albeit horizontal. Although all map projections are compromises between conformality (preservation of angle) and preservation of area, plate carré does not preserve either, meaning it is not particularly well-used.

Hence, \texttt{openstreetmap.org} presents a map in the ``Web Mercator'' projection (also known as Spherical Mercator, WGS 84/Pseudo-Mercator, Google Web Mercator, EPSG:3857). This projection is almost conformal, meaning that (local) angles on the map are the same as angles on the ground, even if lengths are not preserved~\cite{carto-implications-of-webmercator,}. It deviates from normal Mercator in that its calculations are based on the earth being a true sphere, rather than ellipsoidal. The benefit of this is that the calculations for converting between WGS84 latitude/longitude coordinates, and Web Mercator are simpler than normal Mercator~\cite{ugrc-earth-not-round}.

So, to convert from latitude/longitude coordinates to Web Mercator, we can use equations 7--1a \& 7--2a from \textcite[41]{snyder1987map}, where \(\lambda{}\) is longitude and \(\phi{}\) is latitude, both in degrees:

\[x = \frac{\pi R(\lambda^\circ - \lambda^\circ_0)}{180^\circ} \]

\[y = R \ln \tan \left( 45^\circ + \frac{\phi^\circ}{2} \right) \]


\section{Using the HTML5 canvas}

The core of this project is being able to render a map. This, in its most basic form, means being able to take coordinates, which represent a position on the sphere-like body that is the earth, and placing these on a two-dimensional plane. This process is known as map projection~\cite[5]{canters2002small}.

OpenStreetMap uses the WGS84 coordinate reference system (EPSG:4326) to represent the positions of the nodes that make up its data~\cite{osm-wiki-wgs84}. GPS uses this CRS, and it is a popular standard~\cite{epsg.io-epsg:4326}.

If you were to naively place coordinates from the WGS84 system on to a graph, you would get a ``plate carrée'' projected map. This projection is a form of the equirectangular projection, where the standard parallel \(\phi_1 = 0\). In plate carrée, lines of longitude are straight, vertical and equidistant, and lines of latitude are similar, albeit horizontal. Although all map projections are compromises between conformality (preservation of angle) and preservation of area, plate carré does not preserve either, meaning it is not particularly well-used.

Hence, \texttt{openstreetmap.org} presents a map in the ``Web Mercator'' projection (also known as Spherical Mercator, WGS 84/Pseudo-Mercator, Google Web Mercator, EPSG:3857). This projection is almost conformal, meaning that (local) angles on the map are the same as angles on the ground, even if lengths are not preserved~\cite{carto-implications-of-webmercator,}. It deviates from normal Mercator in that its calculations are based on the earth being a true sphere, rather than ellipsoidal. The benefit of this is that the calculations for converting between WGS84 latitude/longitude coordinates, and Web Mercator are simpler than normal Mercator~\cite{ugrc-earth-not-round}.

So, to convert from latitude/longitude coordinates to Web Mercator, we can use equations 7--1a \& 7--2a from \textcite[41]{snyder1987map}, where \(\lambda{}\) is longitude and \(\phi{}\) is latitude, both in degrees:

\[x = \frac{\pi R(\lambda^\circ - \lambda^\circ_0)}{180^\circ} \]

\[y = R \ln \tan \left( 45^\circ + \frac{\phi^\circ}{2} \right) \]


\section{OpenStreetMap data sources}\label{sec:osm-data-sources}

In order to produce a map using OpenStreetMap data, you first need to decide which source/format of data to use. Traditionally, for this project, students use the OpenStreetMap editing API\@. This however, has a number of limitations, some of which I have already discussed in the project plan.

\begin{itemize}
    \item It is designed for map editing applications, and the terms of service explicitly prohibit read-only uses~\cite{osm-owg-tile-usage-policy}.
    \item You can only download a relatively limited geographical area, usually only around 2 km\(^2\). This has the effect that you will not be able to zoom out to see a large area, such an entire country~\cite{osm-wiki-limited-area}.
    \item Complex structures (like buildings with internal courtyards) represented in OpenStreetMap as multipolyons require extensive parsing and validation to correctly display.
\end{itemize}

For online maps, it is common to use either raster or vector ``map tiles''. Raster map tiles are usually 256\(\times{}\)256 \texttt{.png} files, rendered by a Mapnik server. These tiles are named according to their zoom level, and an \(x\) and \(y\) value, where \(x\) and \(y\) are offsets from the top left most tile~\cite{osm-wiki-tile-names}. Raster tiles, however, are not particularly appropriate for offline usage, as you have to request an extremely large amount of tiles, especially when you approach high zoom levels. For example if you downloaded 4 tiles at zoom 10, you would need to download 16 at zoom 11, 64 at zoom 12, and 262144 at zoom 18. This behaviour is unacceptable as rendering tiles is computationally expensive for the OpenStreetMap Foundation, and the size of these tiles adds up.

Vector tiles use a similar scheme, with tiles also being served at Z/X/Y addresses. The key differentiator is that instead of rendered images, vector data is served to clients, and it is up to the clients to render this data into a map. This gives the clients flexibility in how they can choose to display the data, such as the colours, labels, and line thicknesses. This is usually done through stylesheets. As vector data does not pixelate like raster images, you can ``overzoom'' on vector tiles, meaning you do not have to serve tiles to such high zoom levels. One of the most popular formats for vector map tiles are Mapbox Vector tiles. These use the Google Protobuf format to store the vector data.

Vector tiles make a more appropriate offline format than raster tiles, due to their ability to ``overzoom'', and that vector data can be stored more efficiently. Unfortunately it is still not particularly suited to offline usage, as with raster tiles, you need to download many tiles to cover a large region.

Therefore, it makes sense to use a dedicated storage format, that is designed to provide map data for offline applications. The OpenStreetMap Wiki provides a number of possible options for this purpose~\cite{osm-wiki-offline}. The most popular choices are mapsforge~\cite{mapsforge}, which is used by many applications~\cite{apps-using-mapsforge}, and MBTiles (by Mapbox)~\cite{mapbox-mbtiles}. Unfortunately, as the MBTiles format is based on an SQLite database, it can't easily be used in a HTML5 app.

Mapsforge, on the other hand, uses a custom binary data format~\cite{mapsforge-binary-format}. Mapsforge only provide libraries for the Java language and the Android platform. There is partial implementation available of a parser written in JavaScript, which only supports decoding the headers and other metadata. Hence, to use this format for my project, I would need to implement my own parser, which would add substantial complexity. I may also need to tackle issues with memory usage, as parsing and rendering the file will likely require random access within it --- meaning loading the whole file into an \texttt{ArrayBuffer} or some similar structure. Using the \texttt{Blob.slice()} method in JavaScript may allow me to work around this issue.

If parsing the Mapsforge binary files is too complex, I can scale back the project to using the OSM Editing API\@. This will require less parsing work, but will still have plenty of issues to overcome in terms of rendering.


\section{Projecting map data}\label{sec:projecting-map-data}

The core of this project is being able to render a map. This, in its most basic form, means being able to take coordinates, which represent a position on the sphere-like body that is the earth, and placing these on a two-dimensional plane. This process is known as map projection~\cite[5]{canters2002small}.

OpenStreetMap uses the WGS84 coordinate reference system (EPSG:4326) to represent the positions of the nodes that make up its data~\cite{osm-wiki-wgs84}. GPS uses this CRS, and it is a popular standard~\cite{epsg.io-epsg:4326}.

If you were to naively place coordinates from the WGS84 system on to a graph, you would get a ``plate carrée'' projected map. This projection is a form of the equirectangular projection, where the standard parallel \(\phi_1 = 0\). In plate carrée, lines of longitude are straight, vertical and equidistant, and lines of latitude are similar, albeit horizontal. Although all map projections are compromises between conformality (preservation of angle) and preservation of area, plate carré does not preserve either, meaning it is not particularly well-used.

Hence, \texttt{openstreetmap.org} presents a map in the ``Web Mercator'' projection (also known as Spherical Mercator, WGS 84/Pseudo-Mercator, Google Web Mercator, EPSG:3857). This projection is almost conformal, meaning that (local) angles on the map are the same as angles on the ground, even if lengths are not preserved~\cite{carto-implications-of-webmercator,}. It deviates from normal Mercator in that its calculations are based on the earth being a true sphere, rather than ellipsoidal. The benefit of this is that the calculations for converting between WGS84 latitude/longitude coordinates, and Web Mercator are simpler than normal Mercator~\cite{ugrc-earth-not-round}.

So, to convert from latitude/longitude coordinates to Web Mercator, we can use equations 7--1a \& 7--2a from \textcite[41]{snyder1987map}, where \(\lambda{}\) is longitude and \(\phi{}\) is latitude, both in degrees:

\[x = \frac{\pi R(\lambda^\circ - \lambda^\circ_0)}{180^\circ} \]

\[y = R \ln \tan \left( 45^\circ + \frac{\phi^\circ}{2} \right) \]


\clearpage
\chapter{Proof-of-concept development process}

As part of the development process, I have created a number of proof-of-concepts, that each build on each other. These have helped me to understand, test and evaluate various technologies that I could use to build the final application.

\section{Simple offline HTML5 app}

{ \hfill \footnotesize Source in \texttt{proof-of-concepts/1-offline-html5}}

My first proof of concept was a basic offline HTML5 application, employing the techniques that I researched in Section~\ref{sec:offline-html-applications}. Since the \texttt{Application Cache} has been deprecated, the accepted way to create an offline HTML5 app is to use Service Workers~\cite{w3c-service-workers-caches}. The concept behind this is that the Service Worker intercepts all HTTP requests that are made on the website, and can then respond to this request with a result from a prepopulated cache.

In addition to creating an offline HTML5 app, I also learned how to store a large binary blob. This will be necessary for my project, as this is how the map data will be stored. To this end I looked at using both the Service Worker cache, and the File API\@.

The File API works like a traditional file picker --- allowing the user to choose some file from their local filesystem~\cite{w3c-file-api}. This is useful, but does not provide an excellent user experience. For example the user would have to download a map file to some location on their computer, then select it in the file picker every time they want to view a map. This violates Nielsen's ``Recognition, not recall'' heuristic~\cite{nielsen-heuristics}, as the application should work without the user having to remember where they saved the map file.

Therefore, it is preferred to use the Service Worker cache method, where you store the file inside the web browser's cache. You can store binary blobs in the same way as any other website resources:

\begin{lstlisting}[caption=Using the Service Worker cache]
self.addEventListener('install', (event) =>  event.waitUntil(
        addResourcesToCache(['/index.html', '/blob.map']) // prepopulate cache
));

const cacheFirst = async ({ request }) => {
    const responseFromCache = await caches.match(request);
    if (responseFromCache) { // try to get the resource from the cache
        return responseFromCache;
    }

    return new Response('could not retrieve from cache', {
            status: 408,
            headers: { 'Content-Type': 'text/plain' },
        });
};

self.addEventListener('fetch', (event) => event.respondWith(
        cacheFirst({request: event.request})
));
\end{lstlisting}

\section{Rendering geometry to a canvas}

{ \footnotesize Source in \texttt{proof-of-concepts/2-rendering-a-way}, online demo at \href{https://files.george.honeywood.org.uk/2-rendering-a-way/}{\nolinkurl{files.george.honeywood.org.uk/2-rendering-a-way/}}, or \href{https://youtu.be/2F_vpCrQsO4}{demo video}.}

My second proof of concept was to draw some geographical data on to the canvas. As a first step, I had to acquire some data to project. As this was a simple proof of concept I decided to use the GeoJSON format, which is a popular standard for representing geodata, in a fairly human-readable format. The main reason I chose it was that it is easier to parse in JavaScript than the XML data that the OpenStreetMap API provides~\cite{osm-api-wiki}. For my first test, I attempted to render the boundary of Egham, which is currently \href{https://www.openstreetmap.org/way/666914693}{way 666914693}. To convert XML data from OpenStreetMap I used the online \texttt{geojson.io} tool, which allows you to import OSM XML and export GeoJSON~\cite{geojson.io}.

The main challenge I wanted to solve in this proof of concept was projecting the data. This stage is necessary, as OpenStreetMap data comes in the WGS84 CRS, and while this can be naively plotted on a graph as (x,y) coordinates, this would result in the ``plate carrée'' projection, as detailed in Section~\ref{sec:projecting-map-data}. The ``plate carrée'' projection is not particularly desirable, as it does not have any useful properties like conformality (preservation of angle) or preservation of area. Hence, for my application I chose to use Web Mercator, which preserves conformality, and is the de facto standard for web maps.

Initially, I attempted to use the algorithm from \textcite[41]{snyder1987map}, as shown in Section~\ref{sec:projecting-map-data}. Unfortunately I had some issues with getting these equations to produce reasonable results. Therefore, I turned to the OpenStreetMap Wiki, which helpfully provides a reference for transforming WGS84 data into the Web Mercator projection~\cite{osm-wiki-mercator}. I translated the C example into TypeScript, and it correctly projected the data --- see \autoref{lst:projecting} for the implementation.

\begin{lstlisting}[caption=Projecting to Web Mercator, label=lst:projecting]
const RADIANS_TO_DEGREES = 180 / Math.PI;

// calculate x/long
const x = long;
// calculate y/lat
const y = Math.log(Math.tan(
    (lat / RADIANS_TO_DEGREES) / 2 + Math.PI / 4
)) * RADIANS_TO_DEGREES;
\end{lstlisting}

\begin{figure}[ht]
    \centering
    \includegraphics[width=0.7\textwidth]{../proof-of-concepts/2-rendering-a-way/screenshots/ferndown-buildings.png}
    \caption{Rendering some GeoJSON data to the canvas}\label{fig:rendering-geometry}
\end{figure}

This proof of concept was entirely static, with the viewport and zoom level hardcoded into the program. To make the implementation simpler I wrote a function that scaled all the coordinates to be around (0,0), so that I didn't have to handle offsetting the viewport.

\section{Adding interactivity --- panning and zooming}

{ \footnotesize Source in \texttt{proof-of-concepts/3-panning-and-zooming}, online demo at \href{https://files.george.honeywood.org.uk/3-panning-and-zooming/}{\nolinkurl{files.george.honeywood.org.uk/3-panning-and-zooming/}}, or \href{https://youtu.be/_JvGwLra_Q4}{demo video}.}

The third proof of concept heavily built upon this second --- with the goal of adding interactivity in the form of panning and zooming. As being mobile friendly is a priority for my project, I made sure to add touch controls once I had it working with mouse events.

The first part I tackled was the zooming. This turned out to be as simple as multiplying each projected coordinate by some scale factor, which I was already doing in the previous proof of concept. To modify this scale factor, the user can use the +/- buttons, the mouse scroll wheel, or a pinch gesture on a touch device. Pinch gestures were the most challenging to implement, as you have to handle and interpret \texttt{Touch} events for each separate finger on the screen.

Panning also turned out to be relatively easy --- my implementation involved adding some offsets to the projected and scaled coordinates in the latitude and longitude axes. More involved was zooming the map about some arbitrary point. For example, on desktop, you expect a map to zoom into the position of your mouse cursor, when on mobile you expect the middle of your pinch gesture. To achieve this, I update the \texttt{x} and \texttt{y} offsets whenever you zoom, with a smaller offset change when zooming into the top left, and a larger one for the bottom right. Another issue I had to handle is that at a low zoom level, zooming was relatively quick, but once you reached higher zoom levels, it got slower and slower. To compensate for this I made the scale factor logarithmic.

\begin{figure}[ht]
    \centering
    \includegraphics[width=\textwidth]{../proof-of-concepts/3-panning-and-zooming/screenshots/the-world.png}
    \caption{An interactive world map, using Natural Earth data~\cite{natural-earth}}\label{fig:panning-and-zooming}
\end{figure}

For testing the zooming between low and high zoom, I needed some ``large scale'' data to supplement the localized OSM data that I had downloaded for the previous proof of concept. Natural Earth Cultural Vector data seemed suitable, as it is readily available as GeoJSON, is released in the public domain, and is suitably low-detail for my purpose~\cite{natural-earth}. Once again I used \texttt{geojson.io}~\cite{geojson.io}, this time to merge these two sets of data.

To begin with, I was simply redrawing the map every time a \texttt{Touch} or \texttt{Mouse} event fired. This worked, but these events would often fire very rapidly (>60 times/second), resulting in a slow panning experience, where rendering was occurring unnecessarily. My solution for this was \texttt{requestAnimationFrame()}, which allows you to have some function executed at a regular interval by the browser~\cite{mdn-request-animation-frame}. Therefore, I could modify the state of the map as often as necessary, then have it smoothly rendered at regular intervals to reflect any changes to zoom or offsets.

Unfortunately this change meant the canvas re-rendered 60 times a second at all times, even when no state change had occurred, which is a waste of processing power and energy, especially on mobile devices. To remedy this I introduced a \texttt{dirty} flag, which I set whenever a state change had occurred. I then check this \texttt{dirty} flag at the beginning of the \texttt{render()} function, and return early if the internal state is unchanged --- as shown in \autoref{lst:dirty}. The end result was a map that updated smoothly 60 times a second when state changes were occurring, and not at all when the map was static.

\begin{lstlisting}[caption=Only rendering when the map state is \texttt{dirty}, label=lst:dirty]
public render() {
    // if nothing has changed, don't bother re-rendering
    if (!this.dirty) {
        requestAnimationFrame(() => this.render());
        return;
    }
    this.dirty = false;
    [...]
}
\end{lstlisting}

\section{Rendering tiled data from a Mapsforge file}

{ \footnotesize Source in \texttt{proof-of-concepts/4-rendering-osm-data}, online demo at \href{https://files.george.honeywood.org.uk/4-rendering-osm-data/#15/50.7895/-1.8938}{\nolinkurl{files.george.honeywood.org.uk/4-rendering-osm-data/}}, \href{https://youtu.be/2cvZ-veBUto}{mobile demo video}, or \href{https://youtu.be/0evN3RT42QQ}{desktop demo video}.}

My final proof of concept revolved around reading tiled vector OSM data. This allows for a map viewer that is performant at a wide range of zoom levels, as it can switch to more detailed data as you zoom in, and less detailed as you zoom out. It also solves the issue of which data to render --- my previous proof of concepts rendered all the data, all the time, even if it was off canvas. With tiled data, you only have to render the tiles that are currently within the viewport.

The bulk of the work for this proof of concept came in the form of writing a parser for the Mapsforge format~\cite{mapsforge-format}. I decided to use this format based on my report on OpenStreetMap data sources, in Section~\ref{sec:osm-data-sources}. This was because it seemed simpler to parse than MBTiles, which is the other popular choice for tiled vector map files. It is a binary format, that is designed to space-efficiently encode geographical data, such that it can be rendered on low-power mobile devices.

\subsection{The Mapsforge format}

Here I will explain the main concepts of the Mapsforge format, that I had to understand whilst writing the parser. This information is mostly as per the specification~\cite{mapsforge-format}, supplemented by what I learnt as I progressed.

\subsubsection{Points of Interest (PoIs)}

These represent tagged OpenStreetMap nodes, such as shops, bus stops, or any other objects that have been mapped as a node. They are also used for city/town/place labels at low zoom levels.

\subsubsection{Ways}

Ways are an abstraction over OpenStreetMap ways and relations, presenting both as a single type. OSM represents complex structures such as buildings with internal courtyards as multiple ways, for the outer and inner parts. These are then linked by a multipolygon relation that labels each part as inner or outer~\cite{osm-wiki-multipolygon}. These are non-trivial to efficiently interpret, especially with large amounts of data, so it makes sense to push this complexity to the map-writer software.

\subsubsection{Sub-files}

\begin{figure}[ht]
    \centering
    \includegraphics[width=0.8\textwidth]{../proof-of-concepts/4-rendering-osm-data/screenshots/high-detail-at-low-zoom.png}
    \caption{Raw tiled data from zoom 14 base tiles}\label{fig:rendering-tiles}
\end{figure}

%% \- to give latex line break hints

Sub-files store map data for a range of zoom levels. For example, the file \texttt{proof\--of\--concepts/4\--rendering\--osm\--data\-/data/ferndown.map} is compromised of three sub-files, one for z1-z7, based at z5, another for z8-z11, based at z10, and a final one for z12-z21, based at z14. This system provides a compromise between storage space and geographical correctness --- if say we want data for z18, we ``over-zoom'' the z14 sub-file, and vice versa for z1, we ``under-zoom'' the z5 sub-file. Therefore, we don't need to store much duplicated data, compared to storing each zoom level separately.

The over-zoom/under-zoom functionality is supported by each sub-file containing a ``zoom table'', which contains how many PoIs/Ways need to be read for the tile at the requested zoom level. For example, when under-zooming a z14 tile to z12, you will be showing around 112 z14 base tiles (on a 1080p screen). This is too many tiles to be able to render all of their features with acceptable performance. To enable this zoom-table feature, the Way and PoI data in each tile is sorted by ``feature importance'', meaning that data to be shown at low zoom levels, like place labels or important roads, comes first, and less important geometries like houses are placed last. Therefore, you can read that there should be \texttt{x} amount of PoIs/Ways for this tile, for the zoom level you are decoding, read the specified amount, then skip over the rest.

\subsubsection{Variable length integers}

To save file space, coordinates and other numbers are encoded in a custom format. This means that both large and small numbers can use the same representation, whilst requiring a minimal amount of bytes to store them. The format is based around the idea that the first bit of a byte is used as a continuation bit --- i.e., in \texttt{1000 0000} the continuation bit is set. If the continuation bit is set, you continue to read the next byte as part of the same number, until you reach an unset continuation bit.

The format is little-endian, meaning that the first byte of the number is the least significant\footnote{Note that the bytes themselves are still stored in big-endian format}. Hence, you can read the whole number by shifting each of the subsequent numbers by \(7n\), then applying a bitwise OR (represented by \texttt{|} in JavaScript) against the current total. I've provided the commented implementation in \autoref{lst:variable-ints}, as the code is easier to understand than my prose explaination. 

\begin{lstlisting}[caption=Parsing variable length integers, label=lst:variable-ints]
// decode a variable length unsigned integer as a number
getVUint() {
    // if the first bit is 1, need to read the next byte rest of the 7 bits
    // are the numeric value, starting with the least significant
    let value = 0;
    let shift = 0;

    // check if we need to continue
    while ((this.data.getUint8(this.offset) & 0b1000_0000) != 0) {
        // if this not the first byte we've read, each bit is worth more
        value |= (this.data.getUint8(this.offset) & 0b0111_1111) << shift
        this.offset++
        shift += 7
    }

    // read the seven bits from the last byte
    value |= (this.data.getUint8(this.offset) << shift)
    this.offset++
    return value
}
\end{lstlisting}

Signed integers follow a similar scheme with the continuation bit, but also sacrifice the second bit of the last byte to indicate the sign of the number, i.e., \texttt{0100 0000}. There is no concept of a floating point number in the format. Instead, you divide numbers when it is required, for example for coordinates, which are stored in the file in microdegree units.

\subsection{Writing a parser}

In my first iteration of this proof of concept, I was decoding the file inline, using a \texttt{DataView}, manually keeping track of offsets within the blob. This worked, but led to a lot of duplicate code that was just handling offsets, and a number of errors caused by me making some trivial mistake when incrementing the offset by the wrong amount. To remedy this smell, I refactored the code to use the abstraction of a \texttt{Reader}. You construct the \texttt{Reader} with a blob, then call methods like \texttt{.getVUint()}, or \texttt{.getBigUint64()}. These methods increment an offset internal to the \texttt{Reader} class.

\subsection{Rendering the tiles}

\begin{figure}[ht]
    \centering
    \includegraphics[width=\textwidth]{../proof-of-concepts/4-rendering-osm-data/screenshots/styled.png}
    \caption{Styling of data from Mapsforge file}\label{fig:rendering-osm-data}
\end{figure}

When rendering the map, I first convert the current zoom level to the corresponding base zoom level of one of the sub-files. Once I have this I then calculate the coordinates at the top-left and bottom right of the screen, then convert these into Z/X/Y tile numbers, which I can then fetch from the file. To improve rendering performance, only the PoIs and Ways specified in the zoom table for that zoom level are shown.

I handled styling fairly simplistically, setting a number of booleans based on way tags, as the tiles load in. This moves the more complicated (and slow) logic out of the rendering hot path. Inside the \texttt{render()} function I then check these booleans and choose whether a stroke or fill should be used for that way, and what colour it should be. This is shown in Figure~\ref{fig:rendering-osm-data}.

To make sure that the map always draw labels on top, I make two passes through the data. In the first, I render out all the areas and lines, then in the second, the way labels and PoIs. Before I implemented this, areas from other tiles would often overlap labels.

\clearpage
\chapter{Software engineering}\label{sec:software-engineering}

In order to successfully deliver a project, it is important to follow good software engineering practices. These should help you create something that has minimal technical debt, is correct, is maintainable, and can be easily understood by other members of a team.

I decided to write my project in TypeScript over plain JavaScript, as it provides an extremely helpful layer to help solve runtime type issues. It also allows my IDE, Visual Studio Code, to provide far more useful suggestions, compared to writing JS\@. This is because it is aware what types variables are at all times, instead of using a fallible heuristic approach.

Conveniently, the JavaScript bundler I chose, \texttt{esbuild}, has built in support for stripping TypeScript annotations, meaning I only have a single build step. It doesn't have the ability to check types itself, but this was not an issue for me, as VS Code performs this function natively in the editor. \texttt{esbuild} also has a convenient ``dev-server'' mode, where it bundles up the latest code as the requests come in, removing the need for a file-watcher.

\section{Testing}

Testing is critical to producing a piece of software that works as expected. It helps you to define what your function should return before you write it, and ensures that the function returns exactly what you expect.

In my project a significant amount of the work is rendering data to the HTML5 canvas. This portion of the project is very difficult to test --- at least while development is still occurring. Therefore, I directed my testing efforts towards the non-visual portions of the application, such as parsing the Mapsforge file, geometry operations, and projecting data.

I implemented testing using the Jest framework. Initially I had some issues setting it up to be able to understand tests and code written in TypeScript. Once it was set up, it was irritatingly slow, taking 5+ seconds to run with only a single test. The cause of this was two-fold; I was using the \texttt{ts-jest} package, which uses Babel to transpile TypeScript to JS (which is slow), and I had written my \texttt{jest.config.ts} in TypeScript. This meant that when I ran \texttt{jest test}, it had to spin up \texttt{ts-node} to interpret the file, which is also relatively slow. To fix this I rewrote the config file in plain JavaScript, and switched Jest to use \texttt{@swc-node/jest} instead of \texttt{ts-jest} to transpile the JavaScript, which reduced the time taken to under 2 seconds for the whole suite.

\begin{figure}[ht]
    \centering
    \includegraphics[width=0.8\textwidth]{../proof-of-concepts/4-rendering-osm-data/screenshots/artifacts.png}
    \caption{Rendering incorrectly parsed geometries}\label{fig:artifacts}
\end{figure}

In particular the tests for the \texttt{Reader} class proved very useful. At first, I didn't write any formal tests for the \texttt{.getVSint()} method --- instead running it and checking if the output seemed reasonable. This got me so far, with it returning some well-formed data, but when it came to rendering the geometries were all corrupted, as is shown in Figure~\ref{fig:artifacts}. During the debugging process, I wrote exhaustive table tests for this function (see \autoref{lst:vint-test-output}), and I realized that I was incrementing my offset through the file on the wrong line.

These tests made it much easier to refactor the code, without worrying about breaking functionality. This is because I could just run \texttt{pnpm test} to ensure I had not introduced any regressions, and did not need to manually check various different cases.

\begin{lstlisting}[caption=Output from the \texttt{.getVSint()} test suite, numbers=none, label=lst:vint-test-output]
should be able to decode signed variable ints
    pass: 1 byte max negative: [01111111].getVSint() == -63
    pass: 1 byte max positve: [00111111].getVSint() == 63
    pass: 2 byte min: [10000000,00000001].getVSint() == 128 (1 ms)
    pass: 2 byte max: [11111111,01111111].getVSint() == -8191
    pass: 3 byte min: [10000000,10000000,01000001].getVSint() == -16384
    pass: 3 byte max: [11111111,11111111,01111111].getVSint() == -1048575
    pass: 4 byte min: [10000000,10000000,10000000,00000001].getVSint() == 2097152
\end{lstlisting}

\section{Version control}

Another important software engineering practice is making good use of version control. Throughout this project I have made extensive use of Git branches, primarily to develop each of my proof of concepts. This has made sure that the code in the \texttt{main} branch is always in a working state.

I have rigorously followed the \emph{Conventional Commits} specification, prefixing all commits with a type, such as \texttt{feat:}, \texttt{fix:}, \texttt{test:} or \texttt{report:}~\cite{conventional-commits}. This indicates the basic purpose of a commit at a glance, and encourages me to split up large changes into smaller atomic units of a single type. When changes are complicated or have subtle side effects, I have written a longer commit body, to supplement the information in the commit name.

\section{Documentation}

Regarding user documentation, so far my proof of concepts have been simple enough that detailed documentation is not yet necessary. For each proof of concept I have written a \texttt{README.md} that details how it can be run (or a link to it hosted online), and any dependencies or extra files required.

In terms of developer documentation, where appropriate I have written JSDoc comments, particularly for functions that provide a \emph{public} API\@. These explicitly detail the purpose of the function, how it should be used, and what the arguments should be set to. In addition to these JSDoc comments, where the code itself is not completely clear or intentions aren't obvious, I have written comments inline.

\section{Profiling and optimization}\label{sec:profiling}

The \texttt{render()} function is critically important to the responsiveness of the application. As the \texttt{requestAnimationFrame()} callback fires 60 times a second, we should aim to have the \texttt{render()} function run in under \(1000/60=16.666\) ms, so that we have a new frame to show with each refresh of the screen. 

The first tool I implemented to work towards achieving this was adding a frame time metric to the debug information at the top of the canvas. This allowed me to test various different configurations and code changes to see which resulted in the best performance. Measuring the frame time is helpful for gauging performance overall, but it is not very precise --- you cannot see exactly what is consuming the CPU time within the \texttt{render()} call.

\begin{figure}[ht]
    \centering
    \includegraphics[width=\textwidth]{../proof-of-concepts/4-rendering-osm-data/screenshots/profiling.png}
    \caption{Using the profiler}\label{fig:profiling}
\end{figure}

This is where the profiler comes in. It is a tool that lets you visualize how long different function calls within your code take to execute. Both Firefox and Chrome include one within their Developer Tools. I marginally prefer the one Firefox provides (for its Flame Graph, see Figure~\ref{fig:profiling}), but it is important to test in both browsers, as their performance characteristics are surprisingly different. This led me to discover that the vast majority of the time spent in the \texttt{render()} function is in calls on the \texttt{CanvasRenderingContext2D}, to either \texttt{stroke()}, \texttt{fill()}, or draw a \texttt{lineTo()}. Therefore, the most important factor to optimize in this scenario is to reduce the amount of geometries that need to be rendered.

\clearpage
\chapter{Final application development process}

% TODO

\clearpage
\chapter{Conclusion \& evaluation}

\section{Professional issues}

Accessibility is a particularly relevant concern for my project. People with disabilities still need to be able to be able to interpret geographic data, and may have different priorities to able-bodied users. Those with visual impairments need larger map label sizes, and wheelchair users may need extra context about sloped kerbs. My application should accommodate as many of these needs as possible.

Through assistive technologies like screen readers or magnification, most text-based web content is accessible to partially sighted or blind users. Android and iOS both have built-in screen-readers --- TalkBack and VoiceOver respectively. These tools assist users by providing a full alternate means of interaction, that is more complete and usable than simply reading out the text on screen. TalkBack disables the usual touch/swipe gestures on the phone, instead allowing the user to swipe left or right to navigate through tab stops in the interface. At each tab stop TalkBack announces the type and content of the selected screen element, and the user can double-tap to interact with it if necessary. The system navigation gestures can be accessed through two-finger swipes, instead of the usual single finger ones.

Talkback works for not only native apps, but also web based content. This is enabled through use of semantic HTML elements, supplemented by ARIA tagging where this is not possible~\cite{w3c-aria}. The idea behind semantic HTML is that the HTML markup should convey meaning, instead of providing information purely visually through CSS\@. For example, UI buttons should be tagged as \texttt{<button>} elements, over plain \texttt{<div>}s, so that a screen reader can relay this context to the user. There are numerous HTML elements that have semantics, such as basics like \texttt{<ul>} and \texttt{<h1>}, and others that convey more complex concepts, like \texttt{<nav>} for navigation menus~\cite{mdn-accessibility-basis}.

Even most non-text based media can be made accessible to blind people, through \texttt{alt} tags on images and subtitling on videos. Unfortunately this is simply not possible for web maps, as they convey meaning in a visual form that cannot be textually described trivially. MDN advises that in general, web developers should use semantic HTML over the \texttt{<canvas>} element, as the canvas is opaque bitmapped data that cannot be interpreted by a screen reader~\cite{mdn-element-reference-canvas}. Notwithstanding this, blind and visually impaired people still need to navigate the world. \textcite{accessible-wayfinding-belt} found that a ``Tactile Way-finding Belt'', could help facilitate navigation for those with ``visual impairment or Alzheimer’s disease''. They based their study on a system of four vibrating motors, with one for each cardinal direction, which provide signals directing the wearer to their destination. By design this format is more limited than a map, but it does work towards making one of their main uses, navigation, more accessible.

\begin{figure}[ht]
    \centering
    \includegraphics[width=0.7\textwidth]{../final-deliverable/screenshots/simulated-deuteranopia.png}
    \caption{Simulating deuteranopia, in Firefox Developer Tools}\label{fig:deuteranopia}
\end{figure}

A common form of visual impairment is colour vision deficiency (colour blindness). According to the \textcite{nhs-colour-vision-deficiency}, the red-green form affects around 1 in 12 men and 1 in 200 women. This can be more easily accommodated than other forms of visual impairment, by ensuring that text and backgrounds have a suitable contrast ratio, described in the Web Content Accessibility Guidelines~\cite{w3c-wcag}. It is also important to visually check colour contrast through simulators. The Developer Tools in Firefox provide one such simulator, which can visualize various forms of colour vision deficiency --- see \autoref{fig:deuteranopia}.

% TODO: talk about wheelchair access tags in OSM

\begin{figure}[ht]
    \centering
    \includegraphics[width=0.3\textwidth]{images/street-complete-kerb.jpg}
    \caption{Adding kerb height data using the StreetComplete app}\label{fig:sc-kerb}
\end{figure}

Physical disabilities can also be accommodated in an OpenStreetMap-based map. In its simplest form, a point of interest can be tagged with \texttt{wheelchair=(yes|limited|no)}. This tag indicates to what degree an amenity can be accessed by a wheelchair user. \texttt{wheelchair=*} tagging is quite well-used, though not ubiquitous, with around 2.5 million uses worldwide according to Taginfo~\cite{taginfo-wheelchair}. Routing software for wheelchair and walking frame users must avoid features like stairs (\texttt{highway=steps}) and full-height kerbs. Adding this tagging is well-supported in ``on-the-go'' map editing apps like StreetComplete, which has a ``quest'' dedicated to adding this adding kerb details --- see \autoref{fig:sc-kerb}~\cite{streetcomplete}. 

There are specialized map viewers, specifically designed to make viewing wheelchair tagging simple --- most popularly Wheelmap, which uses green, orange, or red to indicate wheelchair accessibility~\cite{wheelmap}. Wheelmap can also be used to add this tagging to amenities, through a simple form, that lowers the barrier of entry for contributing this information to OSM\@. \texttt{smoothness=*} tagging is also valuable to wheelchair users~\cite{osm-wiki-smoothness}. This key has textual values ranging from \texttt{excellent} to \texttt{very\_horrible}, with wheelchair users unlikely to be able to traverse anything at the level of \texttt{smoothness=bad} or worse. Routers can use this information to inform the paths they choose.

One accessibility consideration that I have made whilst developing this project is to include multiple alternative interaction gestures where possible. For example, when zooming the map, you can pick from either the two finger ``pinch'' zoom gesture, or alternatively a single finger tap and hold. This allows people with the ability to use only one hand, such as amputees, or walking stick users, to make full use of the application.

% FIXME: this is all out of date

% In term one I have successfully followed the timeline I set out in Section~\ref{sec:term-1-plan}. During the beginning of the term I fell slightly behind, as I was busy with other modules, but I managed to catch up on lost time. I mildly deviated from the plan in that I added an extra research report. This was on the basics of web technologies, which provides a basis of understanding for the rest of my report, as I had written my other reseasrch assuming quite a high level of base knowledge. My proof of concepts have been useful, helping me to understand key technologies and discover how is best to build the final deliverable.

% In term two I hope to continue keeping up with my plan. I may also write another report, on the subject of WebGL\@. It will be useful to do some research into this technology, as I fear that I have reached the limits of performance of the CPU rendered \texttt{CanvasRenderingContext2D} (see Section~\ref{sec:profiling}). WebGL, which is GPU accelerated, may allow me to go further, without compromising on the level of detail on the map.

% One thing that I have allowed for in the plan for term two is deliverable~\ref{item:deliverable:mobile-testing} (testing on mobile). This may not require as much time as I initially thought, given I have made mobile support a priority throughout the development of the proof of concepts. I also allocated a generous amount of time to integrating my proof of concepts together, which will not be necessary as they have each built on each other.

% The rest of the deliverables planned for term two are all fairly independent (i.e., they do not ``block'' progress). This means that if one is too difficult to implement, I can skip over it and move on to other tasks, providing some flexibility in my schedule. This may be useful for deliverable~\ref{item:deliverable:wikipedia}, where I would provide Wikipedia integration --- as I am unsure whether the Mapsforge files preserve Wikipedia tags.

\clearpage
\chapter{Appendix}

\section{\texttt{diary.md}}

\lstinputlisting[breakindent=0pt, numbers=none, language={}]{../diary.md}

\addcontentsline{toc}{chapter}{Bibliography}

% smaller bibliography text
\renewcommand*{\bibfont}{\normalfont\small}
\printbibliography{}

\end{document}
